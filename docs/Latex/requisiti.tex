\section{Requisiti Funzionali}\label{reqFun}
\subsection{Funzionalità individuali}
\begin{enumerate}[label=\textbf{IF-\arabic*}, ref=IF-\arabic*]

    % LIBRI ---
    \item \textbf{Registrazione di un libro:} \\ Il sistema deve consentire la registrazione di un libro nel catalogo della biblioteca. Durante tale procedura, il sistema deve acquisire tutti i dati definiti nel requisito \ref{df:libro} e presentare all'admin l'interfaccia descritta nel requisito \ref{ui:inserimento}. \label{if1}

    \item \textbf{Modifica dati di un libro registrato:} \\ Il sistema deve consentire la modifica dei dati associati ad un libro precedentemente registrato. Durante tale procedura, il sistema deve acquisire tutti i dati modificabili definiti nel requisito \ref{df:libro} e presentare all'admin l'interfaccia descritta nel requisito \ref{ui:modifica}. \label{if2}

    \item \textbf{Rimozione di un libro:} \\ Il sistema deve consentire la rimozione di un libro precedentemente registrato dal catalogo libri. Durante tale procedura, il sistema deve presentare all'admin l'interfaccia descritta nel requisito \ref{ui:conferma}. \label{if3}

    \item \textbf{Ordinamento libri:} \\ Il sistema deve ordinare lessicograficamente il catalogo dei libri in base al titolo e codice identificativo univoco (ISBN). \label{if4}

    \item \textbf{Visualizzazione del catalogo libri:} \\ Il sistema deve consentire la visualizzazione del catalogo libri precedentemente registrati. Durante tale procedura, il sistema deve mostrare tutti i dati definiti nel requisito \ref{df:libro} e presentare all'admin l'interfaccia descritta nel requisito \ref{ui:principale}. \label{if5}

    \item \textbf{Ricerca libro nel catalogo:} \\ Il sistema deve consentire la ricerca di un libro all'interno del catalogo libri sulla base di titolo, autore o codice identificativo univoco (ISBN). \label{if6}

    \item \textbf{Unicità libro:} \\ Il sistema deve garantire che non siano presenti due libri identici (dunque, aventi lo stesso identificativo secondo \ref{df:libro}) nel catalogo libri. \label{if7}

    \item \textbf{Verifica dati libro:} \\ Il sistema deve verificare che tutti i dati inseriti nei campi specificati nel requisito \ref{df:libro} siano conformi al formato previsto. Non deve essere consentita la presenza nel sistema di un libro i cui dati non rispettino tale formato. \label{if8}
    % ---

    %  Studenti ---
    \item \textbf{Registrazione di uno studente:} \\ Il sistema deve consentire la registrazione di uno studente nel registro della biblioteca. Durante tale procedura, il sistema deve acquisire tutti i dati definiti nel requisito \ref{df:studente} e presentare all'admin l'interfaccia descritta nel requisito \ref{ui:inserimento}. \label{if9}

    \item \textbf{Modifica di uno studente:} \\ Il sistema deve consentire la modifica dei dati associati ad uno studente precedentemente registrato. Durante tale procedura, il sistema deve acquisire tutti i dati modificabili definiti nel requisito \ref{df:studente} e presentare all'admin l'interfaccia descritta nel requisito \ref{ui:modifica}. \label{if10}

    \item \textbf{Rimozione di uno studente:} \\ Il sistema deve consentire la rimozione di uno studente precedentemente registrato dal catalogo studenti. Durante tale procedura, il sistema deve presentare all'admin l'interfaccia descritta nel requisito \ref{ui:conferma}. \label{if11}

    \item \textbf{Ordinamento studenti:} \\ Il sistema deve ordinare lessicograficamente il registro degli studenti in base a cognome, nome e matricola. \label{if12}

    \item \textbf{Visualizzazione del registro studenti:} \\ Il sistema deve consentire la visualizzazione del registro studenti precedentemente registrati. Durante tale procedura, il sistema deve mostrare tutti i dati definiti nel requisito \ref{df:studente} e presentare all'admin l'interfaccia descritta nel requisito \ref{ui:principale}. \label{if13}

    \item \textbf{Ricerca studente nel catalogo:} \\ Il sistema deve consentire la ricerca di uno studente all'interno del registro sulla base di nome, cognome o matricola \label{if14}

    \item \textbf{Unicità studente:} \\ Il sistema deve garantire che non siano presenti due studenti identici (dunque, aventi lo stesso identificativo secondo \ref{df:studente}) nel registro studenti. \label{if15}

    \item \textbf{Verifica dati studente:} \\ Il sistema deve verificare che tutti i dati inseriti nei campi specificati nel requisito \ref{df:studente} siano conformi al formato previsto. Non deve essere consentita la presenza nel sistema di uno studente i cui dati non rispettino tale formato. \label{if16}
    % ----

    % Prestiti ---
    \item \textbf{Registrazione di un prestito:} \\ Il sistema deve consentire la registrazione di un prestito nel registro della biblioteca. Durante tale procedura, il sistema deve acquisire tutti i dati definiti nel requisito \ref{df:prestito} e presentare all'admin l'interfaccia descritta nel requisito \ref{ui:principale}. \label{if17}

    \item \textbf{Modifica di un prestito:} \\ Il sistema deve consentire la modifica dei dati associati ad un prestito precedentemente registrato. Durante tale procedura, il sistema deve acquisire tutti i dati modificabili definiti nel requisito \ref{df:prestito} e presentare all'admin l'interfaccia descritta nel requisito \ref{ui:modifica}. \label{if18}

    \item \textbf{Ordinamento prestiti:} \\ Il sistema deve ordinare cronologicamente l'elenco di prestiti in base alla data di ultima restituzione. \label{if19}

    \item \textbf{Visualizzazione del catalogo prestiti:} \\ Il sistema deve consentire la visualizzazione del catalogo prestiti precedentemente registrati. Durante tale procedura, il sistema deve mostrare tutti i dati definiti nel requisito \ref{df:prestito} e presentare all'admin l'interfaccia descritta nel requisito \ref{ui:principale}. \label{if20}

    \item \textbf{Registrazione della restituzione di un prestito:} \\ Il sistema deve consentire la registrazione della restituzione di un prestito precedentemente registrato. Durante tale procedura, il sistema deve presentare all'admin l'interfaccia descritta nel requisito \ref{ui:conferma}. \label{if21}

    \item \textbf{Rimozione di un prestito: } \\ Il sistema deve consentire la rimozione di un prestito precedentemente registrato dal catalogo prestiti. Durante tale procedura, il sistema deve presentare all'admin l'interfaccia descritta nel requisito \ref{ui:conferma}. \label{if22}

    \item \textbf{Verifica dati prestito:} \\ Il sistema deve verificare che tutti i dati inseriti nei campi specificati nel requisito \ref{df:prestito} siano conformi al formato previsto. Non deve essere consentita la presenza nel sistema di un prestito i cui dati non rispettino tale formato. \label{if23}

    \item \textbf{Verifica limite massimo prestiti:} \\ Il sistema deve garantire che ad un determinato studente non possano essere associati più di 3 prestiti contemporaneamente. \label{if24}

    \item \textbf{Verifica presenza libro:} \\ Il sistema non deve consentire la registrazione di un prestito il cui libro associato sia stato momentaneamente esaurito. \label{if25}

    \item \textbf{Aggiornamento copie disponibili di un libro:} \\ Il sistema deve automaticamente decrementare di 1 il numero di copie di un libro nel momento in cui questo viene associato ad un prestito. Analogamente, il sistema deve incrementare di 1 il numero di copie di un libro precedentemente prestato nel momento in cui esso viene restituito. \label{if26}
    % 

    % Salvataggio --?
    \item \textbf{Salvataggio archivio:} \\ Il sistema deve consentire il salvataggio su file dell'intero database della biblioteca, che comprende: catalogo dei libri, registro degli studenti e registro prestiti. \label{if27}

    \item \textbf{Caricamento archivio:} \\ Il sistema deve consentire l'apertura di un file di salvataggio per caricare il catalogo dei libri, registro degli studenti e registro prestiti in esso contenuti. \label{if28} 

    %Dimenticanza ---
    \item \textbf{Evidenziazione prestiti in ritardo:} \\ Al momento della visualizzazione dei prestiti, il sistema deve evidenziare tutti i prestiti in ritardo per facilitarne l’individuazione da parte dell’admin. \label{if29} 
    

    
\end{enumerate}

\:

\subsection{Esigenze di dati e informazioni}
\begin{enumerate}[label=\textbf{DF-\arabic*}, ref=DF-\arabic*]
    \item \textbf{Dati di un libro:} \label{df:libro} \\
    Il sistema deve memorizzare, per ogni libro registrato, i seguenti dati: \label{df1}
        \begin{itemize}
            \item Titolo \textit{(modificabile)}.
            \item Autori (nome e cognome) \textit{(modificabile)}.
            \item Anno di pubblicazione \textit{(modificabile)}.
            \item Codice identificativo univoco (ISBN) \textit{(identificativo)}.
            \item Numero di copie disponibili \textit{(modificabile)}.
        \end{itemize}

        \item \textbf{Dati di uno studente;} \label{df:studente} \\ 
        Il sistema deve memorizzare, per ogni studente registrato, i seguenti dati: \label{df2}
        \begin{itemize}
            \item Nome \textit{(modificabile)}.
            \item Cognome \textit{(modificabile)}.
            \item Matricola \textit{(identificativo)}.
            \item E-Mail istituzionale (termina in @studenti.unisa.it) \textit{(modificabile)}.
            \item Lista dei libri attualmente in prestito \textit{(modificabile)}.
        \end{itemize}

        \item \textbf{Dati di un prestito:} \label{df:prestito} \\ \label{df3}
        Il sistema deve memorizzare, per ogni prestito registrato, i seguenti dati:
        \begin{itemize}
            \item Studente a cui è associato il prestito.
            \item Libro preso in prestito.
            \item Data ultima di restituzione \textit{(modificabile)}
            \item Restituzione avvenuta o in attesa \textit{(modificabile)}
        \end{itemize}

\end{enumerate}

\subsection{Interfacce con sistemi esterni}
\begin{enumerate}[label=\textbf{IS-\arabic*}, ref=IS-\arabic*]
    \item \textbf{Memorizzazione archivio su file:} \\ \label{is1}
    Il sistema deve salvare su file tutte le informazioni circa:
    \begin{itemize}
        \item Catalogo dei libri.
        \item Registro degli studenti.
        \item Registro dei prestiti.
    \end{itemize}
\end{enumerate}

\:

\subsection{Interfaccia Utente}
\begin{enumerate}[label=\textbf{UI-\arabic*}, ref=UI-\arabic*]
    \item \textbf{Schermata principale.} Deve contenere: \label{ui:principale} \label{ui1}
        \begin{itemize}
            \item Un modo per cambiare la tipologia di informazioni da visualizzare.
            \item Una visualizzazione delle informazioni in forma tabulare.
            \item Un modo per salvare e caricare l'archivio.
            \item Un modo per aggiungere le informazioni della tipologia selezionata \ref{ui:inserimento}.
            \item Un modo per modificare le informazioni della tipologia selezionata\ref{ui:modifica}.
            \item Un modo per eliminare le informazioni della tipologia selezionata\ref{ui:conferma}.
            \item Un modo per cercare le informazioni della tipologia selezionata.
        \end{itemize}

    \item \textbf{Schermata di inserimento dati.} Deve contenere: \label{ui:inserimento} \label{ui2}
        \begin{itemize}
            \item Dei controlli per inserire i dati per la tipologia di informazione selezionata.
            \item Un modo per annullare l'inserimento.
            \item Un modo per confermare l'inserimento.
        \end{itemize}

    \item \textbf{Schermata di modifica dati.} Deve contenere: \label{ui:modifica} \label{ui3}
        \begin{itemize}
            \item Dei controlli per inserire i dati modificabili per la tipologia di informazione selezionata.
            \item Dei controlli in sola lettura per le informazioni non modificabili per la tipologia di informazione selezionata.
            \item Un modo per annullare la modifica.
            \item Un modo per confermare la modifica.
        \end{itemize}

    \item \textbf{Pop-Up di notifica.} Deve contenere: \label{ui:notifica} \label{ui4}
        \begin{itemize}
            \item Un messaggio di notifica.
            \item Un modo per confermare e chiudere la notifica.
        \end{itemize}

    \item \textbf{Pop-Up di conferma.} Deve contenere: \label{ui:conferma} \label{ui5}
        \begin{itemize}
            \item Un'avvertenza.
            \item Un modo per confermare l'operazione.
            \item Un modo per annullare l'operazione.
        \end{itemize}
    
\end{enumerate}
\:
\section{Requisiti non funzionali}
\begin{enumerate}[label=\textbf{FC-\arabic*}, ref=FC-\arabic*]
    \item \textbf{Portabilità:} \\
    Il software deve essere progettato per poter essere eseguito sui sistemi operativi Windows, MacOS e Linux. \label{fc1}

    \item \textbf{Interfaccia responsive:} \\
    Il software ha un'interfaccia che si adatta a varie dimensioni di schermo. \label{fc1}
\end{enumerate}
\clearpage

\section{Tabella di categorizzazione dei requisiti}
\begin{table}[!ht]
\centering
\setlength{\tabcolsep}{6pt}
\renewcommand{\arraystretch}{1.2}
\begin{tabularx}{\textwidth}{|>{\bfseries}l |X| >{\centering\arraybackslash}p{1.6cm} |X|}
\rowcolor{headergray}\bfseries
Prefisso & Requisito & Priorità & Categoria \\ \hline

\ref{if1}  & Registrazione di un libro                              & \cellcolor{priorityAlta}Alta      & \cellcolor{catFunzInd}Funzionalità Individuale \\
\ref{if2}  & Modifica dati di un libro                              & \cellcolor{priorityMedia}Media    & \cellcolor{catFunzInd}Funzionalità Individuale \\
\ref{if3}  & Rimozione di un libro                                  & \cellcolor{priorityAlta}Alta      & \cellcolor{catFunzInd}Funzionalità Individuale \\
\ref{if4}  & Ordinamento libri                                       & \cellcolor{priorityBassa}Bassa   & \cellcolor{catFunzInd}Funzionalità Individuale \\
\ref{if5}  & Visualizzazione catalogo libri                         & \cellcolor{priorityAlta}Alta      & \cellcolor{catFunzInd}Funzionalità Individuale \\
\ref{if6}  & Ricerca libro nel catalogo                             & \cellcolor{priorityMedia}Media    & \cellcolor{catFunzInd}Funzionalità Individuale \\
\ref{if7}  & Unicità libro                                          & \cellcolor{priorityAlta}Alta & \cellcolor{catFunzInd}Funzionalità Individuale \\
\ref{if8}  & Verifica dati libro                                    & \cellcolor{priorityAlta}Alta & \cellcolor{catFunzInd}Funzionalità Individuale \\
\ref{if9}  & Registrazione di uno studente                          & \cellcolor{priorityAlta}Alta  & \cellcolor{catFunzInd}Funzionalità Individuale \\
\ref{if10}  & Modifica di uno studente                              & \cellcolor{priorityMedia}Media & \cellcolor{catFunzInd}Funzionalità Individuale \\
\ref{if11}  & Rimozione di uno studente                             & \cellcolor{priorityAlta}Alta  & \cellcolor{catFunzInd}Funzionalità Individuale \\
\ref{if12} & Ordinamento studenti                                   & \cellcolor{priorityBassa}Bassa & \cellcolor{catFunzInd}Funzionalità Individuale \\
\ref{if13} & Visualizzazione del registro studenti                  & \cellcolor{priorityAlta}Alta  & \cellcolor{catFunzInd}Funzionalità Individuale \\
\ref{if14} & Ricerca studente nel catalogo                          & \cellcolor{priorityMedia}Media & \cellcolor{catFunzInd}Funzionalità Individuale \\
\ref{if15} & Unicità studente                                       & \cellcolor{priorityAlta}Alta & \cellcolor{catFunzInd}Funzionalità Individuale \\
\ref{if16} & Verifica dati studente                                 & \cellcolor{priorityAlta}Alta & \cellcolor{catFunzInd}Funzionalità Individuale \\
\ref{if17} & Registrazione di un prestito                           & \cellcolor{priorityAlta}Alta  & \cellcolor{catFunzInd}Funzionalità Individuale \\
\ref{if18} & Modifica di un prestito                                & \cellcolor{priorityMedia}Media & \cellcolor{catFunzInd}Funzionalità Individuale \\
\ref{if19} & Ordinamento prestiti                                   & \cellcolor{priorityBassa}Bassa & \cellcolor{catFunzInd}Funzionalità Individuale \\
\ref{if20} & Visualizzazione catalogo prestiti                      & \cellcolor{priorityAlta}Alta  & \cellcolor{catFunzInd}Funzionalità Individuale \\
\ref{if21} & Registrazione della restituzione di un prestito        & \cellcolor{priorityMedia}Media & \cellcolor{catFunzInd}Funzionalità Individuale \\
\ref{if22} & Rimozione di un prestito                               & \cellcolor{priorityBassa}Bassa & \cellcolor{catFunzInd}Funzionalità Individuale \\
\ref{if23} & Verifica dati prestito                               & \cellcolor{priorityAlta}Alta & \cellcolor{catFunzInd}Funzionalità Individuale \\
\ref{if24} & Verifica limite massimo prestiti                       & \cellcolor{priorityAlta}Alta & \cellcolor{catFunzInd}Funzionalità Individuale \\
\ref{if25} & Verifica presenza libro                                & \cellcolor{priorityAlta}Alta & \cellcolor{catFunzInd}Funzionalità Individuale \\
\ref{if26} & Aggiornamento copie disponibili                        & \cellcolor{priorityAlta}Alta & \cellcolor{catFunzInd}Funzionalità Individuale \\
\ref{if27} & Salvataggio archivio                                   & \cellcolor{priorityMedia}Media & \cellcolor{catFunzInd}Funzionalità Individuale \\
\ref{if28} & Caricamento archivio                                   & \cellcolor{priorityMedia}Media & \cellcolor{catFunzInd}Funzionalità Individuale \\ \hline
\end{tabularx}
\end{table}

\begin{table}[!ht]
\centering
\setlength{\tabcolsep}{6pt}
\renewcommand{\arraystretch}{1.2}
\begin{tabularx}{\textwidth}{|>{\bfseries}l |X| >{\centering\arraybackslash}p{1.6cm} |X|}
\rowcolor{headergray}\bfseries
Prefisso & Requisito & Priorità & Categoria \\ \hline

\ref{if29} & Evidenziazione prestiti in ritardo                     & \cellcolor{priorityMedia}Media & \cellcolor{catFunzInd}Funzionalità Individuale \\ \hline

\ref{df1}  & Dati di un libro                                       & \cellcolor{priorityAlta}Alta  & \cellcolor{catDatiInfo}Esigenze dati e informazioni \\
\ref{df2}  & Dati di uno studente                                      & \cellcolor{priorityAlta}Alta  & \cellcolor{catDatiInfo}Esigenze dati e informazioni \\
\ref{df3}  & Dati di un prestito                                    & \cellcolor{priorityAlta}Alta  & \cellcolor{catDatiInfo}Esigenze dati e informazioni \\ \hline

\ref{is1}  & Memorizzazione archivio su file                        & \cellcolor{priorityMedia}Media & \cellcolor{catInterfExt}Interfaccia con Sistemi esterni \\ \hline

\ref{ui1}  & Schermata principale                                   & \cellcolor{priorityAlta}Alta  & \cellcolor{catInterfUt}Interfaccia Utente \\
\ref{ui2}  & Schermata di inserimento dati                          & \cellcolor{priorityAlta}Alta  & \cellcolor{catInterfUt}Interfaccia Utente \\
\ref{ui3}  & Schermata di modifica dati                             & \cellcolor{priorityMedia}Media  & \cellcolor{catInterfUt}Interfaccia Utente \\
\ref{ui4}  & Pop-up di notifica                                     & \cellcolor{priorityBassa}Bassa & \cellcolor{catInterfUt}Interfaccia Utente \\
\ref{ui5}  & Pop-up di conferma                                     & \cellcolor{priorityBassa}Bassa & \cellcolor{catInterfUt}Interfaccia Utente \\ \hline
\ref{fc1}  & Portabilità                                            & \cellcolor{priorityAlta}Alta & \cellcolor{catInterfFc}Requisiti non funzionali \\ 
\ref{fc1}  & Interfaccia responsive                                 & \cellcolor{priorityBassa}Bassa & \cellcolor{catInterfFc}Requisiti non funzionali \\ \hline
\end{tabularx}
\end{table}