\section{Principi di buona progettazione}
	\subsection{Progettazione di dettaglio}
		È stata applicata una decomposizione \textbf{Object-Oriented} attorno alle entità del dominio (\textit{Book}, \textit{Student}, \textit{Loan}). Ogni modulo è concepito come un oggetto che incapsula sia lo stato che i comportamento specifici, garantendo un'alta coesione interna. L'architettura sfrutta meccanismi di astrazione come \textit{Interfacce}, \textit{Generics} ed \textit{Ereditarietà}, permettendo di definire logiche di gestione comuni che vengono specializzate per ogni tipo di dato (evidenti nell'interfaccia \verb|Register<T>|).
		
	\subsection{Principio della minima sorpresa}
		Il codice è stato progettato per garantire intuitività e prevedibilità attraverso l'adozione di una \textit{Naming Convention} coerente con gli standard Java, privilegiando identificatori che, sebbene talvolta estesi, risultano autoesplicativi nel definire scopo e responsabilità di classi, metodi e variabili, minimizzando così le ambiguità e facilitando l'immediata comprensione del sistema.
		
	\subsection{Principio della singola responsabilità}
		Il principio è stato applicato a tutti i livelli dell'architettura per garantire alta coesione:
		
		\begin{itemize}
			\item \textbf{Modelli:} Le classi Student, Book e Loan sono \textbf{POJO} (\textit{Plain Old Java Objects}) che hanno l'unica responsabilità di rappresentare i dati e garantirne la validità tramite controlli interni, senza contenere logica di business complessa o riferimenti alla GUI.
			
			\item \textbf{IOManager:} La gestione della persistenza è isolata nella classe \verb|LibraryIOManager|, che si occupa esclusivamente di serializzazione e deserializzazione, rimanendo totalmente disaccoppiata dalla logica operativa della biblioteca.
			
			\item \textbf{Controllers}: La gestione dell'interfaccia è suddivisa per contesti. I RegisterControllers (\textit{Student, Book, Loan}) gestiscono autonomamente i propri tab, mentre i PopupControllers si occupano esclusivamente degli eventi di inserimento/modifica, separando queste operazioni dalla visualizzazione tabellare principale.
		\end{itemize}
		
	\subsection{Principio aperto/chiuso}
		Il sistema è progettato per essere estendibile senza dover modificare il codice esistente:
		
		\begin{itemize}
			\item \textbf{Gerarchia dei Popup:} L'uso di classi astratte (\verb|BookPopupController| ,\\\verb|StudentPopupController|, \verb|LoanPopupController|) definisce una struttura base comune. Se in futuro fosse necessario implementare nuove tipologie di popup (es. per la sola visualizzazione dettagli), basterebbe estendere la classe base senza alterare i controller già funzionanti.
		\end{itemize}
		
	\subsection{Principio di segregazione delle interfacce}
		Per evitare dipendenze inutili, è stato evitato il passaggio di oggetti "pesanti" ai componenti che necessitano solo di funzionalità limitate:
		
		\begin{itemize}
			\item \textbf{Gestione Popup:} Invece di passare l'intero oggetto \verb|Register| ai popup di inserimento (dando accesso a metodi non necessari come la rimozione o la lettura), sono state definite interfacce funzionali specifiche e minimali (\verb|RegisterAdder|, \verb|RegisterModifier|). Questo riduce l'accoppiamento e aumenta la sicurezza del codice.
			
			\item\textbf{Gestione I/O:} La comunicazione tra il \verb|MenuBarController| e il \verb|MainController|, e tra il \verb|LibraryIOManager| e il \verb|MenuBarController|, avviene rispettivamente tramite le interfacce \verb|Refresh| e \verb|ResultActions|, che espongono esclusivamente le operazioni strettamente necessarie (ad esempio l'aggiornamento della vista o la notifica dell'esito di un'operazione di caricamento/salvataggio).
			Questo approccio riduce l'accoppiamento tra i moduli e di conseguenza riduce l'impatto di eventuali modifiche interne favorendo una maggiore manutenibilità del sistema.
		\end{itemize}
		
	\subsection{Principio di inversione delle dipendenze}
		I moduli di alto livello non dipendono dai dettagli implementativi di basso livello:
		
		\begin{itemize}
			\item \textbf{Astrazione dei Registri:} I controller interagiscono con l'interfaccia generica \verb|Register<T>| e non con le implementazioni concrete (es. \verb!StudentRegister!). Questo permette di cambiare l'implementazione sottostante della struttura dati senza impattare sulla logica dell'interfaccia utente.
		\end{itemize}
	
	\subsection{Separazione delle preoccupazioni}
		Il sistema segue una netta divisione logica:
		
		\begin{itemize}
			\item\textbf{Pattern MVC}: L'architettura permette una chiara separazione delle preoccupazioni: i \textbf{Controllers} assumono un ruolo di mediazione: il loro compito è limitato esclusivamente alla gestione dell'interazione tra l'utente e l'interfaccia grafica (\textbf{View}), recependo gli input e invocando le opportune funzioni dei \textbf{Models} senza eseguire logica di business.
			
			\item\textbf{Gestione I/O:} La logica di come salvare i dati è fisicamente separata dalla logica di cosa salvare, mantenendo il core dell'applicazione pulito da dettagli tecnici di I/O.
		\end{itemize}
		
	\subsection{Don't Repeat Yourself (DRY)}
		La duplicazione del codice è stata minimizzata tramite il meccanismo di ereditarietà e di callback:
		\begin{itemize}
			\item \textbf{Logica condivisa nei Controller:} Le operazioni comuni, come la chiusura della finestra o gestione della UI, sono implementate una sola volta nelle classi astratte dei popup e riutilizzate automaticamente dalle sottoclassi di inserimento e modifica.
			
			\item \textbf{Centralizzazione della logica di business nei modelli:} Le operazioni di aggiunta, rimozione e modifica dei dati sono definite esclusivamente all'interno dei modelli, evitando la replicazione della stessa logica in più controller.
			
			\item \textbf{Uso di callback e interfacce funzionali:} Per consentire ai controller di invocare tali operazioni senza duplicarne l'implementazione, i modelli espongono i singoli comportamenti tramite callback incapsulati in interfacce specifiche. I controller ricevono così solo i metodi necessari, mantenendo un chiaro confine tra logica applicativa e logica di presentazione.
		\end{itemize}
		
	\subsection{You Ain't Gonna Need It (YAGNI)}
		Lo sviluppo si è attenuto strettamente ai \hyperref[reqFun]{Requisiti Funzionali} evitando l'introduzione di funzionalità aggiuntive non richieste.
		
		
	\clearpage
	\subsection{Gerarchia dei modelli}
	\begin{figure}[!h]
			\centering
			\includesvg[width=\linewidth]{../Class Diagrams/Detailed View/Diagrams without method parameters/ModelPerspective/ModelsPerspective.svg}
			\caption{Diagramma delle classi per la gerarchia dei modelli.}
	\end{figure}
	
	\clearpage
	\subsection{Prospettiva del MainController}
	\begin{figure}[!h]
		\begin{adjustwidth}{-\oddsidemargin-0.15in}{-\rightmargin}
			\centering
			\includesvg[width=0.8\paperwidth]{../Class Diagrams/Detailed View/Diagrams without method parameters/MainControllerPerspective/MainControllerPerspective.svg}
			\caption{Diagramma delle classi per il MainController}
		\end{adjustwidth}
	\end{figure}

	
	\clearpage
	\subsection{Prospettiva dei RegisterController}
	\begin{figure}[!h]
		\begin{adjustwidth}{-\oddsidemargin-0.8in}{-\rightmargin}
			\centering
			\includesvg[width=0.9\paperwidth]{../Class Diagrams/Detailed View/Diagrams without method parameters/BookRegisterControllerPerspective/BookRegisterControllerPerspective.svg}
			\caption{Diagramma delle classi dalla prospettiva del BookRegisterController}
		\end{adjustwidth}
	\end{figure}
	

	\begin{figure}[!ht]
		\begin{adjustwidth}{-\oddsidemargin-0.9in}{-\rightmargin}
			\centering
			\includesvg[width=0.98\paperwidth]{../Class Diagrams/Detailed View/Diagrams without method parameters/StudentRegisterControllerPerspective/StudentRegisterControllerPerspective.svg}
			\caption{Diagramma delle classi dalla prospettiva del StudentRegisterController}
		\end{adjustwidth}
		
	\end{figure}
	
	\begin{figure}[!h]
		\begin{adjustwidth}{-\oddsidemargin-0.9in}{-\rightmargin}
			\centering
			\includesvg[width=0.98\paperwidth]{../Class Diagrams/Detailed View/Diagrams without method parameters/LoanRegisterControllerPerspective/LoanRegisterControllerPerspective.svg}
			\caption{Diagramma delle classi dalla prospettiva del LoanRegisterController}
		\end{adjustwidth}
	\end{figure}
	
	
	
	\clearpage

\clearpage
\section{Diagrammi di sequenza}

% --- LIBRI ---
\begin{figure}[h]
	\centering
	\includesvg[width=0.9\linewidth]{../Sequence Diagrams/BookInteractions/Add/BookAdd.svg}
	\caption{Diagramma di sequenza per la registrazione di un libro \ref{UC-1}}
\end{figure}

\begin{figure}[!h]
	\centering
	\includesvg[width=0.8\linewidth]{../Sequence Diagrams/BookInteractions/Remove/BookRemove.svg}
	\caption{Diagramma di sequenza per la rimozione di un libro \ref{UC-3}}
\end{figure}

\begin{figure}[h]
	\centering
	\includesvg[width=\linewidth]{../Sequence Diagrams/BookInteractions/Modify/BookModify.svg}
	\caption{Diagramma di sequenza per la modifica di un libro \ref{UC-2}}
\end{figure}


% --------------

% --- STUDENTI ----
\begin{figure}[h]
	\centering
	\includesvg[width=\linewidth]{../Sequence Diagrams/StudentInteractions/Add/StudentAdd.svg}
	\caption{Diagramma di sequenza per la registrazione di uno studente \ref{UC-6}}
\end{figure}

\begin{figure}[h]
	\centering
	\includesvg[width=\linewidth]{../Sequence Diagrams/StudentInteractions/Modify/StudentModify.svg}
	\caption{Diagramma di sequenza per la modifica di uno studente \ref{UC-7}}
\end{figure}

\begin{figure}[h]
	\centering
	\includesvg[width=\linewidth]{../Sequence Diagrams/StudentInteractions/Remove/StudentRemoval.svg}
	\caption{Diagramma di sequenza per la rimozione di uno studente \ref{UC-8}}
\end{figure}
% ---------------

% --- PRESTITI ---
\begin{figure}[h]
	\centering
	\includesvg[width=\linewidth]{../Sequence Diagrams/LoanInteractions/Add/LoanAdd.svg}
	\caption{Diagramma di sequenza per la registrazione di un prestito \ref{UC-11}}
\end{figure}

\begin{figure}[h]
	\centering
	\includesvg[width=\linewidth]{../Sequence Diagrams/LoanInteractions/Modify/LoanModify.svg}
	\caption{Diagramma di sequenza per la modifica di un prestito \ref{UC-12}}
\end{figure}

\begin{figure}[h]
	\centering
	\includesvg[width=\linewidth]{../Sequence Diagrams/LoanInteractions/Remove/LoanRemoval.svg}
	\caption{Diagramma di sequenza per la rimozione di un prestito \ref{UC-13}}
\end{figure}
% -------------

% --- ARCHIVIO ---
\begin{figure}[h]
	\centering
	\includesvg[width=\linewidth]{../Sequence Diagrams/IOInteractions/SaveAs/SaveAs.svg}
	\caption{Diagramma di sequenza per il salvataggio dell'archivio (con nome) \ref{UC-17}}
\end{figure}


\begin{figure}[h]
	\centering
	\includesvg[width=0.85\linewidth]{../Sequence Diagrams/IOInteractions/Save/Save.svg}
	\caption{Diagramma di sequenza per il salvataggio dell'archivio \ref{UC-17}}
\end{figure}

\begin{figure}[h]
	\centering
	\includesvg[width=0.85\linewidth]{../Sequence Diagrams/IOInteractions/Open/Open.svg}
	\caption{Diagramma di sequenza per il caricamento dell'archivio \ref{UC-17}}
\end{figure}

\clearpage
\subsection{Livelli di coesione}
% Tabella di coesione
\begin{table}[H]
	\centering
	\renewcommand{\arraystretch}{1.5} % Aumenta lo spazio tra le righe per leggibilità
	\begin{tabular}{|c|c|p{7cm}|}
		\hline
		\rowcolor{headergray}
		\textbf{\textcolor{black}{Nome classe}} & 
		\textbf{\textcolor{black}{Livello coesione}} & 
		\textbf{\textcolor{black}{Note}} \\ \hline
		
		Library & 
		\cellcolor{priorityBassa}Funzionale & 
		La classe aggrega tutti i register per fornire un punto di accesso unico tramite le funzioni getter. \\ \hline
		
		Student, Book e Loan &
		\cellcolor{priorityBassa}Funzionale & Contiene tutti gli attributi e metodi per la visione e gestione di una singola entità. \\ \hline
		
		\makecell{Student, Book e Loan \\Register} &
		\cellcolor{priorityBassa}Funzionale &
		Le classi implementano un registro del tipo specificato, e forniscono i metodi utili alla gestione di quest'ultimo. \\ \hline
		
		LibraryIOManager &
		\cellcolor{priorityMedia}Comunicazionale & 
		La classe implementa le funzionalità utili all'interfacciamento della biblioteca con un file esterno di memorizzazione.  \\ \hline
		
		MainController &
		\cellcolor{catDatiInfo}Temporale &La classe implementa le funzionalità utili all'inizializzazione dei principali controller e del gestore IO. \\ \hline
		
		MenuBarController &
		\cellcolor{priorityBassa}Funzionale &
		La classe gestisce il menu bar fornendo l'accesso alle funzione di IO tramite la classe LibraryIOManager \\ \hline
		
		\makecell{Student, Book e Loan \\RegisterController} &
		\cellcolor{priorityMedia}Comunicazionale &
		Permette la visualizzazione dei dati in tabella, implementa le funzioni di ricerca e rimozione. Interfaccia il registro con i Popup di inserimento e modifica. \\ \hline
		
		\makecell{Student, Book e Loan \\InsertPopupController} &
		\cellcolor{priorityBassa}Funzionale &
		La classe implementa le funzionalità utili all'interfacciamento dell'admin con il registro al fine di consentire l'inserimento di un elemento del tipo specificato.\\ \hline
		
		\makecell{Student, Book e Loan \\ModifyPopupController} &
		\cellcolor{priorityBassa}Funzionale &
		La classe implementa le funzionalità utili all'interfacciamento dell'admin con il registro al fine di consentire la modifica di un elemento del tipo specificato. \\ \hline
		
		SideBarController &
		\cellcolor{priorityMedia}Comunicazionale & La classe implementa le funzionalità utili per l'accesso alle funzionalità di gestione del registro e alla ricerca  \\ \hline
	
		
	\end{tabular}
\end{table}


\clearpage
\subsection{Livelli di accoppiamento}
% Tabella di accoppiamento

\begin{table}[h]
	\centering
	\renewcommand{\arraystretch}{1.5} % Aumenta lo spazio tra le righe per leggibilità
	\begin{tabular}{|c|c|c|}
		\hline
		\rowcolor{headergray}
		\textbf{Classe 1} & 
		\textbf{Classe 2} & 
		\textbf{Accoppiamento} \\ \hline
		
		Loan & Student &
		\cellcolor{priorityMedia}Per timbro \\ \hline
		
		Loan & Book &
		\cellcolor{priorityMedia}Per timbro \\ \hline
		
		Student & StudentRegister &
		\cellcolor{priorityBassa}Per dati \\ \hline
		
		Book & BookRegister &
		\cellcolor{priorityBassa}Per dati \\ \hline
		
		Loan & LoanRegister &
		\cellcolor{priorityBassa}Per dati \\ \hline
		
		Library & Register &
		\cellcolor{priorityBassa} Per dati  \\ \hline
		
		MainController & MenuBarController &
		\cellcolor{priorityMedia}Per timbro \\ \hline
		
		*RegisterController & MainController &
		\cellcolor{priorityBassa}Per dati  \\ \hline
		
		MenuBarController & LibraryIOManager &
		\cellcolor{priorityMedia}Per timbro \\ \hline
		
		*RegisterController & SideBarController &
		\cellcolor{priorityBassa}Per dati  \\ \hline
		
		*RegisterController & *InsertPopupController &
		\cellcolor{priorityBassa}Per dati \\ \hline
		
		*RegisterController & *ModifyPopupController &
		\cellcolor{priorityBassa}Per dati  \\ \hline
		
	\end{tabular}
\end{table}

L'accoppiamento per timbro tra \verb|Loan| e \verb|Student/Book| sussiste in quanto al Loan non sono necessarie tutte le funzionalità fornite dalle due classi. Sebbene fosse possibile ridurre l'accoppiamento mediante introduzione di opportune interfacce ciò avrebbe complicato la stampa del registro prestiti. Si è dunque ritenuto questo accoppiamento accettabile. \\

L'accoppiamento per timbro tra \verb|MainController| e \verb|MenuBarController| sussiste in quanto il medesimo riceve dal \verb|MainController| una \verb|Library| che gli dà accesso a più funzionalità di quante ne sono effettivamente richieste. Per le stesse motivazioni sussiste un accoppiamento per timbro tra \verb|MenuBarController| e \verb|LibraryIOManager| (la struttura dati passata è la stessa).
%% \subsection{}
