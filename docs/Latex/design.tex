\section{Diagrammi delle classi}
	\subsection{Gerarchia dei modelli}
	
	\begin{figure}[H]
		\begin{adjustwidth}{-\oddsidemargin-0.9in}{-\rightmargin}
			\centering
			\includesvg[width=0.98\paperwidth]{../Class Diagrams/Model hierarchy/ModelsHierarchy.svg}
			\caption{Diagramma delle classi per la gerarchia dei modelli.}
		\end{adjustwidth}
	\end{figure}
	
	\clearpage
	\subsection{Prospettiva del MainController}
	\begin{figure}[H]
		\begin{adjustwidth}{-\oddsidemargin-0.9in}{-\rightmargin}
			\centering
			\includesvg[width=0.98\paperwidth]{../Class Diagrams/Main Controller Perspective/MainControllerPerspective.svg}
			\caption{Diagramma delle classi per il MainController}
		\end{adjustwidth}
	\end{figure}

	
	\clearpage
	\subsection{Prospettiva dei RegisterController}
	
	\begin{figure}[!ht]
		\begin{adjustwidth}{-\oddsidemargin-0.9in}{-\rightmargin}
			\centering
			\includesvg[width=0.98\paperwidth]{../Class Diagrams/Register Controllers Perspective/Book/BookControllerPerspective.svg}
			\caption{Diagramma delle classi dalla prospettiva del BookRegisterController}
		\end{adjustwidth}
	\end{figure}
	
	\begin{figure}[!ht]
		\begin{adjustwidth}{-\oddsidemargin-0.9in}{-\rightmargin}
			\centering
			\includesvg[width=0.98\paperwidth]{../Class Diagrams/Register Controllers Perspective/Student/StudentControllerPerspective.svg}
			\caption{Diagramma delle classi dalla prospettiva del StudentRegisterController}
		\end{adjustwidth}
		
	\end{figure}
	
	\begin{figure}[!ht]
		\begin{adjustwidth}{-\oddsidemargin-0.9in}{-\rightmargin}
			\centering
			\includesvg[width=0.98\paperwidth]{../Class Diagrams/Register Controllers Perspective/Loan/LoanControllerPerspective.svg}
			\caption{Diagramma delle classi dalla prospettiva del LoanRegisterController}
		\end{adjustwidth}
	\end{figure}
	
	
	\clearpage

\clearpage
\section{Diagrammi di sequenza}

% --- LIBRI ---
\begin{figure}[h]
	\centering
	\includesvg[width=\linewidth]{../Sequence Diagrams/BookInteractions/Add/BookAdd.svg}
	\caption{Diagramma di sequenza per la registrazione di un libro \ref{UC-1}}
\end{figure}

\begin{figure}[H]
	\centering
	\includesvg[width=\linewidth]{../Sequence Diagrams/BookInteractions/Modify/BookModify.svg}
	\caption{Diagramma di sequenza per la modifica di un libro \ref{UC-2}}
\end{figure}

\begin{figure}[h]
	\centering
	\includesvg[width=\linewidth]{../Sequence Diagrams/BookInteractions/Remove/BookRemove.svg}
	\caption{Diagramma di sequenza per la rimozione di un libro \ref{UC-3}}
\end{figure}
% --------------

% --- STUDENTI ----
\begin{figure}[h]
	\centering
	\includesvg[width=\linewidth]{../Sequence Diagrams/StudentInteractions/Add/StudentAdd.svg}
	\caption{Diagramma di sequenza per la registrazione di uno studente \ref{UC-6}}
\end{figure}

\begin{figure}[h]
	\centering
	\includesvg[width=\linewidth]{../Sequence Diagrams/StudentInteractions/Modify/StudentModify.svg}
	\caption{Diagramma di sequenza per la modifica di uno studente \ref{UC-7}}
\end{figure}

\begin{figure}[h]
	\centering
	\includesvg[width=\linewidth]{../Sequence Diagrams/StudentInteractions/Remove/StudentRemoval.svg}
	\caption{Diagramma di sequenza per la rimozione di uno studente \ref{UC-8}}
\end{figure}
% ---------------

% --- PRESTITI ---
\begin{figure}[h]
	\centering
	\includesvg[width=\linewidth]{../Sequence Diagrams/LoanInteractions/Add/LoanAdd.svg}
	\caption{Diagramma di sequenza per la registrazione di un prestito \ref{UC-11}}
\end{figure}

\begin{figure}[h]
	\centering
	\includesvg[width=\linewidth]{../Sequence Diagrams/LoanInteractions/Modify/LoanModify.svg}
	\caption{Diagramma di sequenza per la modifica di un prestito \ref{UC-12}}
\end{figure}

\begin{figure}[h]
	\centering
	\includesvg[width=\linewidth]{../Sequence Diagrams/LoanInteractions/Remove/LoanRemoval.svg}
	\caption{Diagramma di sequenza per la rimozione di un prestito \ref{UC-13}}
\end{figure}
% -------------



\clearpage
\subsection{Livelli di coesione}
% Tabella di coesione
\begin{table}[H]
	\centering
	\renewcommand{\arraystretch}{1.5} % Aumenta lo spazio tra le righe per leggibilità
	\begin{tabular}{|c|c|p{7cm}|}
		\hline
		\rowcolor{headergray}
		\textbf{\textcolor{black}{Nome classe}} & 
		\textbf{\textcolor{black}{Livello coesione}} & 
		\textbf{\textcolor{black}{Note}} \\ \hline
		
		Library & 
		\cellcolor{priorityBassa}Funzionale & 
		La classe aggrega tutti i register per fornire un punto di accesso unico tramite le funzioni getter. \\ \hline
		
		Student, Book e Loan &
		\cellcolor{priorityBassa}Funzionale & Contiene tutti gli attributi e metodi per la visione e gestione di una singola entità. \\ \hline
		
		\makecell{Student, Book e Loan \\Register} &
		\cellcolor{priorityBassa}Funzionale &
		Le classi implementano un registro del tipo specificato, e forniscono i metodi utili alla gestione di quest'ultimo. \\ \hline
		
		LibraryIOManager &
		\cellcolor{priorityMedia}Comunicazionale & 
		La classe implementa le funzionalità utili all'interfacciamento della biblioteca con un file esterno di memorizzazione.  \\ \hline
		
		MainController &
		\cellcolor{catDatiInfo}Temporale &La classe implementa le funzionalità utili all'inizializzazione dei principali controller e del gestore IO. \\ \hline
		
		MenuBarController &
		\cellcolor{priorityBassa}Funzionale &
		La classe gestisce il menu bar fornendo l'accesso alle funzione di IO tramite la classe LibraryIOManager \\ \hline
		
		\makecell{Student, Book e Loan \\RegisterController} &
		\cellcolor{priorityMedia}Comunicazionale &
		Permette la visualizzazione dei dati in tabella, implementa le funzioni di ricerca e rimozione. Interfaccia il registro con i Popup di inserimento e modifica. \\ \hline
		
		\makecell{Student, Book e Loan \\InsertPopupController} &
		\cellcolor{priorityBassa}Funzionale &
		La classe implementa le funzionalità utili all'interfacciamento dell'admin con il registro al fine di consentire l'inserimento di un elemento del tipo specificato.\\ \hline
		
		\makecell{Student, Book e Loan \\ModifyPopupController} &
		\cellcolor{priorityBassa}Funzionale &
		La classe implementa le funzionalità utili all'interfacciamento dell'admin con il registro al fine di consentire la modifica di un elemento del tipo specificato. \\ \hline
		
		SideBarController &
		\cellcolor{priorityMedia}Comunicazionale & La classe implementa le funzionalità utili per l'accesso alle funzionalità di gestione del registro e alla ricerca  \\ \hline
	
		
	\end{tabular}
\end{table}


\clearpage
\subsection{Livelli di accoppiamento}
% Tabella di accoppiamento

\begin{table}[h]
	\centering
	\renewcommand{\arraystretch}{1.5} % Aumenta lo spazio tra le righe per leggibilità
	\begin{tabular}{|c|c|c|}
		\hline
		\rowcolor{headergray}
		\textbf{Classe 1} & 
		\textbf{Classe 2} & 
		\textbf{Accoppiamento} \\ \hline
		
		Loan & Student &
		\cellcolor{priorityBassa}Per dati \\ \hline
		
		Loan & Book &
		\cellcolor{priorityBassa}Per dati \\ \hline
		
		Student & StudentRegister &
		\cellcolor{priorityBassa}Per dati \\ \hline
		
		Book & BookRegister &
		\cellcolor{priorityBassa}Per dati \\ \hline
		
		Loan & LoanRegister &
		\cellcolor{priorityBassa}Per dati \\ \hline
		
		Library & Register &
		\cellcolor{priorityBassa} Per dati  \\ \hline
		
		MainController & MenuBarController &
		\cellcolor{priorityMedia}Per timbro \\ \hline
		
		*RegisterController & MainController &
		\cellcolor{priorityBassa}Per dati  \\ \hline
		
		MenuBarController & LibraryIOManager &
		\cellcolor{priorityMedia}Per timbro \\ \hline
		
		*RegisterController & SideBarController &
		\cellcolor{priorityBassa}Per dati  \\ \hline
		
		*RegisterController & *InsertPopupController &
		\cellcolor{priorityBassa}Per dati \\ \hline
		
		*RegisterController & *ModifyPopupController &
		\cellcolor{priorityBassa}Per dati  \\ \hline
		
	\end{tabular}
\end{table}


%% \subsection{}

\section{Principi di buona progettazione}

\subsection{Principio della minima sorpresa}
Nel design del sistema è stata data priorità alla chiarezza della nomenclatura. Sono stati prediletti identificatori per classi e metodi che, sebbene talvolta lunghi, risultano esplicativi del loro scopo e della loro responsabilità, minimizzando l'ambiguità. Inoltre, è stata mantenuta una rigida coerenza nelle convenzioni di denominazione (naming convention) lungo tutto il progetto: ad esempio, le classi di registro seguono lo schema EntitàRegister (es. StudentRegister) così come i relativi controller (es. StudentRegisterController).

\subsection{Principio di segregazione delle interfacce}
Questo principio trova specifica applicazione nella gestione dei Pop-up di modifica e inserimento. Attraverso l'utilizzo di interfacce funzionali, i controller dei Pop-up sono stati disaccoppiati dall'intera interfaccia Register. Ad essi vengono passati esclusivamente i riferimenti ai metodi strettamente necessari per l'operazione.

\subsection{Privilegiare l'associazione all'ereditarietà}
Si è scelto di limitare l'ereditarietà esclusivamente all'estensione di classi astratte per la condivisione di codice strutturale, preferendo la composizione per le interazioni tra componenti. 


\subsection{Ortogonalità}
L'indipendenza tra i componenti è stata perseguita su due livelli. Sull'asse orizzontale, la gestione della persistenza (LibraryIOManager) è stata isolata dalla logica di business: questo garantisce che eventuali modifiche alle modalità di salvataggio su file non generino effetti collaterali sul funzionamento dell'applicazione. Sull'asse verticale, i domini Book e Student operano come compartimenti isolati; è possibile evolvere la gestione anagrafica degli studenti senza alcun impatto sulla gestione dei libri. Tuttavia, l'ortogonalità del sistema non è assoluta a causa dell'entità Loan. Essendo per sua natura un'entità relazionale, il Prestito crea un punto di accoppiamento strutturale: modifiche sostanziali ai modelli Book o Student rischiano inevitabilmente di propagarsi alla logica di gestione dei prestiti, rendendo questa componente meno indipendente rispetto alle altre.


\subsection{DRY - Don't Repeat Yourself}
Per evitare la duplicazione del codice e favorire la manutenibilità, è stata definita una classe astratta per i Pop-Up contenente le definizioni di attributi e metodi comuni, evitando di ripeterli nelle implementazioni specifiche. Inoltre, l'adozione dell'interfaccia comune Register ha permesso di uniformare le operazioni sui diversi tipi di dati.

\subsection{YANGI - You Ain't Gonna Need It}
Lo sviluppo si è attenuto strettamente ai \ref{Requisitifunzionali} evitando l'introduzione di funzionalità speculative non richieste.