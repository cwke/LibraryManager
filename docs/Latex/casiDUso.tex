\iffalse % Si è un commento...
%
\item  
\textbf{Titolo:}  \\
\textbf{Attori:} Admin\\\\
\textbf{Precondizioni:} L'admin si trova nella schermata principale e ha selezionato la tipologia di elemento \\
\textbf{Flusso principale:} 
\begin{enumerate}[label=\arabic*.]
	\item 
	\item
	\item 
\end{enumerate} \\
\textbf{Post-condizioni:} \\\\
\textbf{Flusso alternativo 1:}
\begin{enumerate}
	\item[4] 
\end{enumerate}
%
\par\noindent\rule{\textwidth}{0.5pt}

\fi

\begin{enumerate}[label=\textbf{UC-\arabic*}, ref=UC-\arabic*]

    %
    \item \label{UC-1}
    \textbf{Titolo:} Registrazione libro\\
    \textbf{Attori:} Admin \\\\
    \textbf{Precondizioni:} L'admin si trova nella schermata di visualizzazione della tipologia di elemento "libro".\\
    \textbf{Flusso principale: } 
        \begin{enumerate}[label=\arabic*.]
            \item L'admin avvia la funzione di inserimento.
            \item L'admin compila i campi specificati in \ref{df:libro}.
            \item L'admin conferma l'operazione
        \end{enumerate} 
    \textbf{Post-condizioni:} Il libro è correttamente inserito nell'elenco libri. \\\\
    \textbf{Flusso alternativo 1:}
        \begin{enumerate}
            \item[4] I dati inseriti sono errati. L'admin deve inserirne di nuovi. Si riparte dal punto 2.
        \end{enumerate}
    \textbf{Flusso alternativo 2:}
        \begin{enumerate}
            \item[3] L'admin annulla l'operazione o chiude la finestra.
        \end{enumerate}
    \textbf{Flusso alternativo 3:}
        \begin{enumerate}
            \item[4] È già presente un libro avente lo stesso identificativo. Si riparte dal punto 2. 
        \end{enumerate}
    %
    \par\noindent\rule{\textwidth}{0.5pt}
    %
    \item \label{UC-2}
    \textbf{Titolo:} Modifica libro \\
    \textbf{Attori:} Admin\\\\
    \textbf{Precondizioni:} L'admin si trova nella schermata di visualizzazione della tipologia di elemento "libro".\\
    \textbf{Flusso principale:} 
        \begin{enumerate}[label=\arabic*.]
            \item L'admin seleziona un libro da modificare.
            \item L'admin avvia la funzione di modifica.
            \item L'admin compila i campi (modificabili) specificati in \ref{df:libro}.
            \item L'admin conferma l'operazione.
        \end{enumerate} 
    \textbf{Post-condizioni:} Il libro è correttamente modificato.\\\\
    \textbf{Flusso alternativo 1:}
        \begin{enumerate}
            \item[5] I dati inseriti sono errati. L'admin deve inserire di nuovi. Si riparte dal punto 2. 
        \end{enumerate}
    \textbf{Flusso alternativo 2:}
        \begin{enumerate}
            \item[3] L'admin annulla l'operazione o chiude la finestra.
        \end{enumerate}
    %
    \par\noindent\rule{\textwidth}{0.5pt}
    %
    \item \label{UC-3}
    \textbf{Titolo:} Rimozione libro \\
    \textbf{Attori:} Admin \\\\
    \textbf{Precondizioni:}L'admin si trova nella schermata di visualizzazione della tipologia di elemento "libro". \\
    \textbf{Flusso principale:} 
        \begin{enumerate}[label=\arabic*.]
            \item L'admin seleziona un libro da rimuovere.
            \item L'admin avvia la funzione di rimozione.
            \item L'admin conferma la rimozione.
        \end{enumerate} 
    \textbf{Post-condizioni:} Il libro è correttamente rimosso dall'elenco libri.\\\\
    \textbf{Flusso alternativo 1:}
        \begin{enumerate}
            \item[3] L'admin annulla l'operazione o chiude la finestra 
        \end{enumerate}
    %
    \par\noindent\rule{\textwidth}{0.5pt}
    %
    \item \label{UC-4}
    \textbf{Titolo:} Visualizzazione libri \\
    \textbf{Attori:} Admin \\\\
    \textbf{Precondizioni:} L'admin si trova in una schermata di visualizzazione. \\
    \textbf{Flusso principale:} 
        \begin{enumerate}[label=\arabic*.]
            \item L'admin accede alla schermata di visualizzazione selezionando la tipologia di elemento "libro".
            \item L'admin scorre i libri della schermata di visualizzazione.
        \end{enumerate} 
    \textbf{Post-condizioni:} L'admin visualizza i libri registrati
    %
    \par\noindent\rule{\textwidth}{0.5pt}
    %
    \item \label{UC-5}
    \textbf{Titolo:} Ricerca titolo\\
    \textbf{Attori:} Admin\\\\
    \textbf{Precondizioni:} L'admin si trova nella schermata di visualizzazione della tipologia di elemento "libro".\\
    \textbf{Flusso principale:} 
        \begin{enumerate}[label=\arabic*.]
            \item L'admin scrive del testo all'interno della barra di ricerca
        \end{enumerate} 
    \textbf{Post-condizioni:} Il sistema mostra dei risultati sulla base delle corrispondenze trovate rispetto al testo inserito nella barra di ricerca.
    %
    \par\noindent\rule{\textwidth}{0.5pt}
    %
    \item \label{UC-6}
    \textbf{Titolo:} Registrazione studente \\
    \textbf{Attori:} Admin\\\\
    \textbf{Precondizioni:}L'admin si trova nella schermata di visualizzazione della tipologia di elemento "studente".\\
    \textbf{Flusso principale:} 
        \begin{enumerate}[label=\arabic*.]
            \item L'admin avvia la funzione di inserimento.
            \item L'admin compila i campi specificati in \ref{df:studente}, eccetto "Lista dei libri attualmente in prestito".
            \item L'admin conferma l'operazione.
        \end{enumerate} 
    \textbf{Post-condizioni:} Lo studente è correttamente inserito nell'elenco studenti.\\\\
    \textbf{Flusso alternativo 1:}
        \begin{enumerate}
            \item[3] I dati inseriti sono errati. L'admin deve inserirne di nuovi. Si riparte dal punto 2. 
        \end{enumerate}
    \textbf{Flusso alternativo 2:}
        \begin{enumerate}
            \item[3] L'admin annulla l'operazione o chiude la finestra.
        \end{enumerate}
    \textbf{Flusso alternativo 3:}
        \begin{enumerate}
            \item[4] È già presente uno studente avente stesso identificativo. Si riparte dal punto 2.
        \end{enumerate}
    %
    \par\noindent\rule{\textwidth}{0.5pt}
    %
    \item \label{UC-7}
    \textbf{Titolo:} Modifica studente\\
    \textbf{Attori:} Admin\\\\
    \textbf{Precondizioni: }L'admin si trova nella schermata di visualizzazione della tipologia di elemento "studente".\\
    \textbf{Flusso principale:} 
        \begin{enumerate}[label=\arabic*.]
            \item L'admin seleziona uno studente da modificare.
            \item L'admin avvia la funzione di modifica.
            \item L'admin compila i campi (modificabili) specificati in \ref{df:studente}, eccetto "Lista dei libri attualmente in prestito".
            \item L'admin conferma l'operazione
        \end{enumerate} 
    \textbf{Post-condizioni:} Lo studente è correttamente modificato.\\\\
    \textbf{Flusso alternativo 1:}
        \begin{enumerate}
            \item[5] I dati sono errati. L'admin deve inserirne di nuovi. Si riparte dal punto 2. 
        \end{enumerate}
    \textbf{Flusso alternativo 2:}
        \begin{enumerate}
            \item[4] L'admin annulla l'operazione o chiude la finestra.
        \end{enumerate}
    %
    \par\noindent\rule{\textwidth}{0.5pt}
    %
    \item \label{UC-8}
    \textbf{Titolo:} Rimozione studente\\
    \textbf{Attori:} Admin\\\\
    \textbf{Precondizioni:} L'admin si trova nella schermata di visualizzazione della tipologia di elemento "studente".\\
    \textbf{Flusso principale:} 
        \begin{enumerate}[label=\arabic*.]
            \item L'admin seleziona uno studente da rimuovere.
            \item L'admin avvia la funzione di rimozione
            \item L'admin conferma la rimozione
        \end{enumerate} 
    \textbf{Post-condizioni:} Lo studente è correttamente rimosso dall'elenco studenti\\\\
    \textbf{Flusso alternativo 1:}
        \begin{enumerate}
            \item[4] L'admin annulla l'operazione o chiude la finestra.
        \end{enumerate}
    %
    \par\noindent\rule{\textwidth}{0.5pt}
    %
    \item \label{UC-9}
    \textbf{Titolo:} Visualizzazione studenti\\
    \textbf{Attori:} Admin\\\\
    \textbf{Precondizioni:}L'admin si trova in una schermata di visualizzazione.\\
    \textbf{Flusso principale:} 
        \begin{enumerate}[label=\arabic*.]
            \item L'admin accede alla schermata di visualizzazione.
            \item L'admin scorre gli studenti nella schermata di visualizzazione
        \end{enumerate} 
    \textbf{Post-condizioni:} L'admin visualizza gli studenti registrati.
    %
    \par\noindent\rule{\textwidth}{0.5pt}
    %
    \item \label{UC-10}
    \textbf{Titolo:} Ricerca studente\\
    \textbf{Attori:} Admin\\\\
    \textbf{Precondizioni:} L'admin si trova nella schermata di visualizzazione della tipologia di elemento "studente".\\
    \textbf{Flusso principale:} 
        \begin{enumerate}[label=\arabic*.]
            \item L'admin scrive del testo all'interno della barra di ricerca.
        \end{enumerate} 
    \textbf{Post-condizioni:} Il sistema mostra dei risultati sulla base delle corrispondenze trovate rispetto al testo inserito nella barra di ricerca.
    %
    \par\noindent\rule{\textwidth}{0.5pt}
    %
    \item \label{UC-11}
    \textbf{Titolo:} Registrazione prestito\\
    \textbf{Attori:} Admin\\\\
    \textbf{Precondizioni:}L'admin si trova nella schermata di visualizzazione della tipologia di elemento "prestito".\\
    \textbf{Flusso principale:} 
        \begin{enumerate}[label=\arabic*.]
            \item L'admin avvia la funzione di inserimento.
            \item L'admin compila i campi specificati in \ref{df:prestito}, eccetto "Restituzione avvenuta o in attesa".
            \item L'admin conferma l'operazione.
        \end{enumerate} 
    \textbf{Post-condizioni:} Il prestito è correttamente inserito nell'elenco prestiti.\\\\
    \textbf{Flusso alternativo 1:}
        \begin{enumerate}
            \item[4] I dati inseriti sono errati. L'admin deve inserirne di nuovi. Si riparte dal punto 2. 
        \end{enumerate}
    \textbf{Flusso alternativo 2:}
        \begin{enumerate}
            \item[3] L'admin annulla l'operazione o chiude la finestra.
        \end{enumerate}
    \textbf{Flusso alternativo 3:}
        \begin{enumerate}
            \item[4] Lo studente specificato ha già 3 prestiti non ancora estinti. Si riparte dal punto 2.
        \end{enumerate}
    \textbf{Flusso alternativo 4:}
        \begin{enumerate}
            \item[4] Il libro specificato è esaurito (nessuna copia rimanente). Si riparte dal punto 2.
        \end{enumerate}
    %
    \par\noindent\rule{\textwidth}{0.5pt}
    %
    \item \label{UC-12}
    \textbf{Titolo:} Modifica prestito\\
    \textbf{Attori:} Admin\\\\
    \textbf{Precondizioni:}L'admin si trova nella schermata di visualizzazione della tipologia di elemento "prestito".\\
    \textbf{Flusso principale:} 
        \begin{enumerate}[label=\arabic*.]
            \item L'admin seleziona un prestito da modificare.
            \item L'admin avvia la funzione di modifica.
            \item L'admin modifica "Data ultima di restituzione"
            \item L'admin conferma l'operazione.
        \end{enumerate} 
    \textbf{Post-condizioni:} Il prestito è correttamente modificato.\\\\
    \textbf{Flusso alternativo 1:}
        \begin{enumerate}
            \item[5] I dati inseriti sono errati. L'admin deve inserirne di nuovi. Si riparte dal punto 2. 
        \end{enumerate}

    \textbf{Flusso alternativo 2:}
        \begin{enumerate}
            \item[5] L'admin annulla l'operazione o chiude la finestra.
        \end{enumerate}
    %
    \par\noindent\rule{\textwidth}{0.5pt}
    %
    \item \label{UC-13}
    \textbf{Titolo:} Rimozione prestito\\
    \textbf{Attori:} Admin\\\\
    \textbf{Precondizioni:} L'admin si trova nella schermata di visualizzazione della tipologia di elemento "prestito".\\
    \textbf{Flusso principale:} 
        \begin{enumerate}[label=\arabic*.]
            \item L'admin seleziona un prestito da rimuovere.
            \item L'admin avvia la funzione di rimozione.
            \item L'admin conferma la rimozione.
        \end{enumerate} 
    \textbf{Post-condizioni:} Il prestito è correttamente rimosso dall'elenco prestiti.\\\\
    \textbf{Flusso alternativo 1:}
        \begin{enumerate}
            \item[3] L'admin annulla l'operazione o chiude la finestra. 
        \end{enumerate}
    %
    \par\noindent\rule{\textwidth}{0.5pt}
    %
    \item \label{UC-14}
    \textbf{Titolo:} Visualizzazione prestiti\\
    \textbf{Attori:} Admin\\\\
    \textbf{Precondizioni:} L'admin si trova nella schermata di una visualizzazione.\\
    \textbf{Flusso principale:} 
        \begin{enumerate}[label=\arabic*.]
            \item L'admin accede alla schermata di visualizzazione.
            \item L'admin scorre i prestiti nella schermata di visualizzazione
        \end{enumerate} 
    \textbf{Post-condizioni:} L'admin visualizza i prestiti registrati.
    %
    \par\noindent\rule{\textwidth}{0.5pt}
    %
    \item \label{UC-15}
    \textbf{Titolo:}  Ricerca prestito\\
    \textbf{Attori:} Admin\\\\
    \textbf{Precondizioni:} L'admin si trova nella schermata di visualizzazione della tipologia di elemento "prestito".\\
    \textbf{Flusso principale:} 
        \begin{enumerate}[label=\arabic*.]
            \item L'admin scrive del testo all'interno della barra di ricerca.
        \end{enumerate} 
    \textbf{Post-condizioni:} Il sistema mostra dei risultati sulla base delle corrispondenze trovate rispetto al testo inserito nella barra di ricerca.
    %
    \par\noindent\rule{\textwidth}{0.5pt}
    %
    \item \label{UC-16}
    \textbf{Titolo:}  Estinzione prestito\\
    \textbf{Attori:} Admin\\\\
    \textbf{Precondizioni:} L'admin si trova nella schermata di visualizzazione della tipologia di elemento "prestito".\\
    \textbf{Flusso principale:} 
        \begin{enumerate}[label=\arabic*.]
            \item L'admin seleziona un prestito da estinguere.
            \item L'admin avvia la funzione di estinzione.
            \item L'admin conferma l'operazione.
        \end{enumerate} 
    \textbf{Post-condizioni:} Il prestito risulta correttamente contrassegnato come estinto.\\\\
    \textbf{Flusso alternativo 1:}
        \begin{enumerate}
            \item[3] L'admin annulla l'operazione o chiude la finestra. 
        \end{enumerate}
    \textbf{Flusso alternativo 2:}
        \begin{enumerate}
            \item[2] Il prestito è già stato estinto. L'operazione è disabilitata. 
        \end{enumerate}
    %
    \par\noindent\rule{\textwidth}{0.5pt}
    %
    \item \label{UC-17}
    \textbf{Titolo:} Salvataggio archivio\\
    \textbf{Attori:} Admin\\\\
    \textbf{Precondizioni:} L'admin si trova in una schermata di visualizzazione.\\
    \textbf{Flusso principale:} 
        \begin{enumerate}[label=\arabic*.]
            \item L'admin accede alla schermata di salvataggio.
            \item L'admin avvia l'operazione di salvataggio.
            \item L'admin conferma l'operazione.
        \end{enumerate} 
    \textbf{Post-condizioni:} L'archivio è correttamente salvato su file.\\\\
    \textbf{Flusso alternativo 1:}
        \begin{enumerate}
            \item[1] L'admin annulla l'operazione o chiude la finestra.
        \end{enumerate}
    %
    \par\noindent\rule{\textwidth}{0.5pt}
    %
    \item \label{UC-18}
    \textbf{Titolo:} Caricamento archivio\\
    \textbf{Attori:} Admin\\\\
    \textbf{Precondizioni:} L'admin si trova in una schermata di visualizzazione.\\
    \textbf{Flusso principale:} 
        \begin{enumerate}[label=\arabic*.]
            \item L'admin accede alla schermata di caricamento file
            \item L'admin seleziona il file contenente la biblioteca da caricare.
            \item L'admin conferma l'operazione
        \end{enumerate} 
    \textbf{Post-condizioni:} La biblioteca i cui dati erano memorizzati nel file selezionato viene correttamente caricata.\\\\
    \textbf{Flusso alternativo 1:}
        \begin{enumerate}
            \item[4] L'estensione del file selezionato è diversa da quella prevista. L'operazione è annullata: si ritorna al punto 1. 
        \end{enumerate}
    \textbf{Flusso alternativo 2:}
        \begin{enumerate}
            \item[4] Il formato del file selezionato è diverso da quello previsto. L'operazione è annullata si ritorna al punto 1.
        \end{enumerate}
    \textbf{Flusso alternativo 3:}
        \begin{enumerate}
            \item[3] L'admin annulla l'operazione o chiude la finestra.
        \end{enumerate}
    %
    \par\noindent\rule{\textwidth}{0.5pt}
    
\end{enumerate}
\clearpage
\section{Diagrammi dei casi d'uso}

\begin{figure}[!ht]
	\begin{adjustwidth}{-\oddsidemargin-0.9in}{-\rightmargin}
		\centering
		\includesvg[width=0.98\paperwidth]{../Use Case Diagrams/svg/BookDiagram.svg}
		\caption{\small Diagramma dei casi d'uso: gestione dei libri (\ref{UC-1}, \ref{UC-2}, \ref{UC-3}, \ref{UC-4}, \ref{UC-5}).}
	\end{adjustwidth}
\end{figure}

\begin{figure}[H]
	\begin{adjustwidth}{-\oddsidemargin-0.9in}{-\rightmargin}
		\centering
		\includesvg[width=0.98\paperwidth]{../Use Case Diagrams/svg/StudentDiagram.svg}
		\caption{\footnotesize Diagramma dei casi d'uso: gestione degli studenti (\ref{UC-6}, \ref{UC-7}, \ref{UC-8}, \ref{UC-9}, \ref{UC-10}).}
	\end{adjustwidth}
\end{figure}

\begin{figure}[!ht]
	\begin{adjustwidth}{-\oddsidemargin-0.9in}{-\rightmargin}
		\centering
		\includesvg[width=0.98\paperwidth]{../Use Case Diagrams/svg/LoanDiagram.svg}
		\caption{ Diagramma dei casi d'uso: gestione dei prestiti (\ref{UC-11}, \ref{UC-12}, \ref{UC-13}, \ref{UC-14}, \ref{UC-15}, \ref{UC-16}).}
	\end{adjustwidth}
\end{figure}

\begin{figure}[h]
	\centering
	\includesvg[width=0.6\linewidth]{../Use Case Diagrams/svg/IODiagram.svg}
	\caption{Diagramma dei casi d'uso: gestione dell'I/O (\ref{UC-17}, \ref{UC-18}).}
\end{figure}